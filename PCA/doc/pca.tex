\input{../../tex_functions/functions.tex}

\begin{document}
\footnotesize

\title{Principle component analysis}
\author{Evan Cummings\\
CSCI 548 -- Douglas W.~Raiford -- Pattern Recognition}

\maketitle
\begin{multicols*}{2}

\section{\texttt{class} command :}

The class of the \texttt{iris} dataset is a \texttt{data.frame}, a ``tightly coupled collections of variables which share many of the properties of matrices and of lists, used as the fundamental data structure by most of R's modeling software.''

\begin{figure}[H]
  \centering
    \includegraphics[width=0.45\textwidth]{images/class_command.png}
\end{figure}

\section{\texttt{summary} command :}

The \texttt{summary} command applied to the \texttt{iris} dataset, is a ``generic function used to produce result summaries of the results of various model fitting functions.  The function invokes particular methods which depend on the `class' of the first argument.''

\begin{figure}[H]
  \centering
    \includegraphics[width=0.45\textwidth]{images/summary_command.png}
\end{figure}

\section{\texttt{labels} command :}

The \texttt{labels} command: ``Find[s] a suitable set of labels from an object for use in printing or plotting.  For example, a generic function.''

\begin{figure}[H]
  \centering
    \includegraphics[width=0.45\textwidth]{images/labels_command.png}
\end{figure}

\section{\texttt{colnames} command :}

The \texttt{colnames} command: ``Retrieve[s] or set[s] the row or column names of a matrix-like object.''

\begin{figure}[H]
  \centering
    \includegraphics[width=0.45\textwidth]{images/colnames_command.png}
\end{figure}

\section{tab-completion :}

Pressing \texttt{tab} after a \texttt{data.frame} object with accessor-operator \texttt{\$} will provide variable names.

\begin{figure}[H]
  \centering
    \includegraphics[width=0.45\textwidth]{images/tab_completion.png}
\end{figure}

\section{Iris data PCA plotting :}

\begin{figure}[H]
  \centering
    \includegraphics[width=0.45\textwidth]{images/prcomp_tab.png}
\end{figure}

\begin{figure}[H]
  \centering
    \includegraphics[width=0.45\textwidth]{images/pca_plot.png}
  \caption{PCA-plot showing what appears to be two clusters, despite the fact that there are three distinct species of iris.}
\end{figure}

The \texttt{plot} command, a ``Generic function for plotting of R objects.  For more details about the graphical parameter arguments, see `par'.

For simple scatter plots, \texttt{plot.default} will be used.  However, there are \texttt{plot} methods for many R objects, including \texttt{function}'s, \texttt{data.frame}'s, \texttt{density} objects, etc.  Use \texttt{methods(plot)} and the documentation for these.''

\begin{figure}[H]
  \centering
    \includegraphics[width=0.45\textwidth]{images/pca_plot_red.png}
\end{figure}

\begin{figure}[H]
  \centering
    \includegraphics[width=0.35\textwidth]{images/pca_plot_class.png}
\end{figure}

\begin{figure}[H]
  \centering
    \includegraphics[width=0.45\textwidth]{images/pca_pvar_barplot.png}
\end{figure}

\subsection{Source code :}

\begin{Rs}
# store the iris 'classes' :
c = iris[,5]

# store the iris 'data' :
d = iris[,seq(1,4)]

# perform PCA on d :
p = prcomp(d)

# set the 1st principle component :
x = p$x[,1]

# set the 2nd principle component :
y = p$x[,2]

# plot them :
png('../doc/images/pca_plot.png')
plot(x,y)
dev.off()

# plot them in red :
png('../doc/images/pca_plot_red.png')
plot(x,y, col='red', bg='red', type='p', pch=21)
dev.off()

# store indexes of classes :
s  = which(c == 'setosa')
vc = which(c == 'versicolor')
vg = which(c == 'virginica')

# color vector :
cc     = as.vector(c)
cc[s]  = 'red'
cc[vc] = 'blue'
cc[vg] = 'green'

# plot them colored by class :
png('../doc/images/pca_plot_class.png')
X11(width=4, height=4)
par(mar=c(2.5,2.5,2.5,.1),mgp = c(1.5, .5, 0))
plot(x,y, col=cc, bg=cc, type='p', pch=21, xlab='First Principal Component',
     ylab='Second Principal Component', main='Iris Data')
legend('topright', levels(c),
       col   = c('red', 'green', 'blue'), 
       pt.bg = c('red', 'green', 'blue'), pch = 21)
dev.off()

# do a barplot of the proportions of variance :
sig  = p$sdev**2
pvar =  sig / sum(sig)
png('../doc/images/pca_pvar_barplot.png')
barplot(pvar, names.arg=c('PC1','PC2','PC3','PC4'),
        main='Proportions of Variance')
dev.off()
\end{Rs}

\section{Fruit data :}

While the iris data appears to be highly grouped, the fruit data appears much less so.  However, the lemons seem to be clustered distintly from the other fruit.  Therefore, I believe that these data \emph{do} show enough structure for machine learning techniques; with apples, oranges, and peaches potentially presenting the highest challenge.

\begin{figure}[H]
  \centering
    \includegraphics[width=0.45\textwidth]{images/pca_plot_fruit.png}
\end{figure}

\begin{figure}[H]
  \centering
    \includegraphics[width=0.45\textwidth]{images/pca_pvar_fruit_barplot.png}
  \caption{The proportions of the variance in the fruit data.  The principle component captures $\approx$ 94\% of the variation.}
\end{figure}

\begin{figure}[H]
  \centering
    \includegraphics[width=0.45\textwidth]{images/fruit_3d.png}
  \caption{In 3D, we can see much more distinct clustering.}
\end{figure}

\begin{figure}[H]
  \centering
    \includegraphics[width=0.45\textwidth]{images/pca_plot_fruit_23.png}
  \caption{The second and third principle components, highlighting the clustering nature of the fruit data.}
\end{figure}

\subsection{Source code :}
\begin{Rs}
# read the variable back in :
f = read.csv("../data/fruit.csv")

# store the fruit 'classes' :
c = f[,5]

# store the fruit 'data' :
d = f[,seq(1,4)]

# perform PCA on d :
p = prcomp(d)

# set the 1st principle component :
x = p$x[,1]

# set the 2nd principle component :
y = p$x[,2]

# set the 3nd principle component :
z = p$x[,3]

# store indexes of classes :
a = which(c == 'apple')
l = which(c == 'lemon')
o = which(c == 'orange')
z = which(c == 'peach')

# color vector :
cols  = c('red', 'yellow', 'orange', 'burlywood1')
cc    = as.vector(c)
cc[a] = cols[1]
cc[l] = cols[2]
cc[o] = cols[3]
cc[z] = cols[4]

# plot them colored by class :
png('../doc/images/pca_plot_fruit.png')
#X11(width=4, height=4)
par(mar=c(2.5,2.5,2.5,.1),mgp = c(1.5, .5, 0))
plot(x,y, col=cc, bg=cc, type='p', pch=21, xlab='First Principal Component',
     ylab='Second Principal Component', main='Fruit Data')
legend('topright', levels(c), col = cols, pt.bg = cols, pch = 21)
dev.off()

# do a barplot of the proportions of variance :
sig  = p$sdev**2
pvar =  sig / sum(sig)
png('../doc/images/pca_pvar_fruit_barplot.png')
barplot(pvar, names.arg=c('PC1','PC2','PC3','PC4'),
        main='Proportions of Variance')
dev.off()

# import the rgl library for 3D stuff:
library(rgl)

plot3d(x,y,z, col=cc, bg=cc, type='p', pch=21,
       xlab='First Principal Component',
       ylab='Second Principal Component',
       zlab='Third Principal Component')

# plot them colored by class :
png('../doc/images/pca_plot_fruit_23.png')
#X11(width=4, height=4)
par(mar=c(2.5,2.5,2.5,.1),mgp = c(1.5, .5, 0))
plot(y,z, col=cc, bg=cc, type='p', pch=21, xlab='Second Principal Component',
     ylab='Third Principal Component', main='Fruit Data')
legend('topright', levels(c), col = cols, pt.bg = cols, pch = 21)
dev.off()
\end{Rs}


\end{multicols*}

%\begin{figure}[H]
%  \centering
%    \includegraphics[width=0.45\textwidth]{images/.png}
%\end{figure}


\end{document}


