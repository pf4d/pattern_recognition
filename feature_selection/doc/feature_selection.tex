\input{../../tex_functions/functions.tex}

\begin{document}
\footnotesize

\title{Feature selection}
\author{Evan Cummings\\
CSCI 548 -- Douglas W.~Raiford -- Pattern Recognition}

\maketitle

\section{iris data}

\begin{figure}[H]
  %\centering
  \begin{minipage}[b]{0.45\linewidth}
    \includegraphics[width=\linewidth]{images/iris_ttest.pdf}
  \end{minipage}
  \hfill
  \quad
  \begin{minipage}[b]{0.45\linewidth}
    \centering
    \begin{tabular}[b]{c|lll}
      & \textbf{full} & \textbf{$t$-test} & \textbf{seq-fwrd} \\
      \hline
      rand.~acc.   & 97.4\% & 97.4\% & 97.4\% \\ 
      $n$-way acc. & 98\%   & 96\%   & 96\% \\
      dims         & all    & 3,4    & 1,3 \\
      $J_1$        & 6.50   & 15.2   & 7.41 \\
      $J_2$        & 41.7   & 22.4   & 24.6 \\
      $J_3$        & 31.8   & 19.4   & 22.9 \\
    \end{tabular}
    \vspace{5mm}
  \end{minipage}
  \hfill
  \vspace{5mm}
  \quad
  \begin{minipage}[t]{1.00\linewidth}
    \includegraphics[width=\linewidth]{images/iris.pdf}
  \end{minipage}
  \caption{The \texttt{iris} data projected onto the first two linear discriminants using: a.) all the dimensions; b.) two best dimensions resulting from the relative-$t$-test (top left barplot); and c.), two best dimensions determined from the sequential-forward algorithm.  The random 75\% training data score, leave-one-out-cross-validation score, best dimension indicies, and $J$-scores are provided in the above right table for the full model, $t$-test-best-dimensions model, and sequential-forward-best-dimensions model.   Note that while both the $t$-test and sequential-forward models resulted in identical $n$-way-cross-validation scores, the $J_2$ and $J_3$ values associated with the sequential-forward model are slightly improved.}
\end{figure}

\section{fruit data}

\begin{figure}[H]
  %\centering
  \begin{minipage}[b]{0.45\linewidth}
    \includegraphics[width=\linewidth]{images/fruit_ttest.pdf}
  \end{minipage}
  \hfill
  \quad
  \begin{minipage}[b]{0.45\linewidth}
    \centering
    \begin{tabular}[b]{c|lll}
      & \textbf{full} & \textbf{$t$-test} & \textbf{seq-fwrd} \\
      \hline
      rand.~acc.   & 96.4\% & 95.6\%  & 95.6\% \\ 
      $n$-way acc. & 95.1\% & 94.1\%  & 94.1\% \\
      dims         & all    & 1,2     & 1,2 \\
      $J_1$        & 0.19   & 9.53    & 9.53 \\
      $J_2$        & 33.2   & 21.0    & 21.0 \\
      $J_3$        & 21.7   & 20.0    & 20.0 \\
    \end{tabular}
    \vspace{5mm}
  \end{minipage}
  \hfill
  \vspace{5mm}
  \quad
  \begin{minipage}[t]{1.00\linewidth}
    \includegraphics[width=\linewidth]{images/fruit.pdf}
  \end{minipage}
  \caption{The \texttt{fruit} data projected onto the first two linear discriminants using: a.) all the dimensions; b.) two best dimensions resulting from the relative-$t$-test (top left barplot); and c.), two best dimensions determined from the sequential-forward algorithm, in this case identical to those of the $t$-test.  The random 75\% training data score, leave-one-out-cross-validation score, best dimension indicies, and $J$-scores are provided in the above right table for the full model, $t$-test-best-dimensions model, and sequential-forward-best-dimensions model.  Note that the reduced-dimensional model performed slightly worse than the full-dimensional model.}
\end{figure}

\section{tumor data}

\begin{figure}[H]
  %\centering
  \begin{minipage}[b]{0.45\linewidth}
    \includegraphics[width=\linewidth]{images/tumor_ttest.pdf}
  \end{minipage}
  \hfill
  \quad
  \begin{minipage}[b]{0.45\linewidth}
    \centering
    \begin{tabular}[b]{c|lll}
      & \textbf{full} & \textbf{$t$-test} & \textbf{seq-fwrd} \\
      \hline
      rand.~acc.   & 96\%   & 93.14\% & 97.1\% \\ 
      $n$-way acc. & 95.7\% & 94.7\%  & 95.1\% \\
      dims         & all    & 2,3,6   & 1,3,6 \\
      $J_1$        & 1.30   & 2.01    & 1.67 \\
      $J_2$        & 6.11   & 4.96    & 5.23 \\
      $J_3$        & 5.11   & 3.96    & 4.23 \\
    \end{tabular}
    \vspace{5mm}
  \end{minipage}
  \hfill
  \vspace{5mm}
  \quad
  \begin{minipage}[t]{1.00\linewidth}
    \includegraphics[width=\linewidth]{images/tumor.pdf}
  \end{minipage}
  \caption{The density of the \texttt{tumor} data projected onto the first linear discriminant using: a.) all the dimensions; b.) three best dimensions resulting from the relative-$t$-test (top left barplot); and c.), three best dimensions determined from the sequential-forward algorithm.  The random 75\% training data score, leave-one-out-cross-validation score, best dimension indicies, and $J$-scores are provided in the above right table for the full model, $t$-test-best-dimensions model, and sequential-forward-best-dimensions model.   Note that the sequential-forward-derived dimensions performed slightly better than the $t$-test-derived dimensions, despite having a lower $J_1$ score.}
\end{figure}

\section{mouse data}

\begin{figure}[H]
  %\centering
  \begin{minipage}[b]{0.65\linewidth}
    \hspace{5mm}
    \includegraphics[width=\linewidth]{images/mouse_ttest.pdf}
  \vspace{10mm}
  \end{minipage}
  \hfill
  \quad
  \begin{minipage}[b]{0.45\linewidth}
    \centering
    \begin{tabular}[b]{c|c|c}
      \textbf{dims} & \textbf{accuracy} & \textbf{method} \\
      \hline
      5   & 75\% & t-test \\
      12  & 85\% & seq-fwrd\\
      13  & 80\% & seq-fwrd\\
      14  & 70\% & seq-fwrd\\
      15  & 65\% & seq-fwrd\\
    \end{tabular}
    \vspace{20mm}
  \end{minipage}
  \hfill
  \vspace{10mm}
  \quad
  \begin{minipage}[t]{1.00\linewidth}
    \includegraphics[width=\linewidth]{images/mouse.pdf}
  \end{minipage}
  \caption{The density of the \texttt{mouse} data projected onto the first linear discriminant using the five-best dimensions from the relative-$t$-test (top left barplot).  The table (upper right) gives the performance for the sequential-forward $n$-best dimensions.  Note that as the number of dimensions increases, the $n$-way-cross-validation scores decrease.} 
\end{figure}

\section{fertility dataset}

\begin{figure}[H]
  %\centering
  \begin{minipage}[b]{0.45\linewidth}
    \includegraphics[width=\linewidth]{images/fertility_ttest.pdf}
  \end{minipage}
  \hfill
  \quad
  \begin{minipage}[b]{0.45\linewidth}
    \centering
    \begin{tabular}[b]{c|lll}
      & \textbf{full} & \textbf{$t$-test} & \textbf{seq-fwrd} \\
      \hline
      rand.~acc.   & 92\%                        & 92\%       & 92\% \\ 
      $n$-way acc. & 85\%                        & 86\%       & 86\% \\
      dims         & all                         & 1,2,4,6,7  & 1,2,4,6,7 \\
      $J_1$        & $1.57 \times 10\sups{-2}$   & $2.73 \times 10\sups{-2}$ & $2.73 \times 10\sups{-2}$ \\
      $J_2$        & 1.13  & 1.12       & 1.12 \\
      $J_3$        & 0.132 & 0.117      & 0.117 \\
    \end{tabular}
    \vspace{5mm}
  \end{minipage}
  \hfill
  \vspace{5mm}
  \quad
  \begin{minipage}[t]{1.00\linewidth}
    \includegraphics[width=\linewidth]{images/fertility.pdf}
  \end{minipage}
  \caption{The density of the \texttt{fertility} data projected onto the first linear discriminant using: a.) all the dimensions; b.) five best dimensions resulting from the relative-$t$-test (top left barplot); and c.), five best dimensions determined from the sequential-forward algorithm -- indentical to the $t$-test-derived dimensions.  The random 75\% training data score, leave-one-out-cross-validation score, best dimension indicies, and $J$-scores are provided in the above right table for the full model, $t$-test-best-dimensions model, and sequential-forward-best-dimensions model.   Note that reduced-dimensional models performed slightly better than the full-dimensional model.}
\end{figure}

\newpage

\section{source code}

\begin{multicols}{2}

\subsection{functions}

\Rexternal{../src/functs.r}

\subsection{iris data}

\Rexternal{../src/iris.r}

\subsection{fruit data}

\Rexternal{../src/fruit.r}

\subsection{tumor data}

\Rexternal{../src/tumor.r}

\subsection{mouse data}

\Rexternal{../src/mouse.r}

\subsection{fertility data}

\Rexternal{../src/fertility.r}


\end{multicols}

%\begin{figure}[H]
%  \centering
%    \includegraphics[width=\linewidth]{images/.png}
%\end{figure}


\end{document}


